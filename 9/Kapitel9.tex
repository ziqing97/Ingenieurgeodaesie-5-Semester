\documentclass[12pt]{article}
\usepackage{setspace,graphicx,amsmath,geometry,fontspec,titlesec,soul,bm,subfigure}
\titleformat{\section}[block]{\LARGE\bfseries}{\arabic{section}}{1em}{}[]
\titleformat{\subsection}[block]{\Large\bfseries\mdseries}{\arabic{section}.\arabic{subsection}}{1em}{}[]
\titleformat{\subsubsection}[block]{\normalsize\bfseries}{\arabic{subsection}-\alph{subsubsection}}{1em}{}[]
\titleformat{\paragraph}[block]{\small\bfseries}{[\arabic{paragraph}]}{1em}{}[]
\setmainfont{Times New Roman}
\renewcommand{\baselinestretch}{1.15}
\renewcommand\contentsname{Inhaltverzeichnis}
\geometry{a4paper,left=2.5cm,right=2.5cm,top=2.5cm,bottom=2.5cm}
\begin{document}
	\newpagestyle{main}{            
		\sethead{}{Kapitel 6}{} 
		\setfoot{}{\thepage}{}
		\headrule
		\footrule
			}
	\pagestyle{main}
\tableofcontents
\newpage
\section{Terrestrisches Laserscanning}	
\subsection{Grundlagen und Systemübersicht}
\begin{itemize}
	\item Winkelmessung über Encoder, gleichabständige Tastung. 
	\item Streckenmessung im Impuls- oder Phasenvergleichsverfahren
	\item Strhachskompensator und/oder Horizontier- (und Zentrier) rorichtung (nicht Zwangsweise notwendig)
	\item Intensität der reflektierten Signals als vierte Messgrößer $\Rightarrow$ Keine Messung von Einzelpunkten
	\item Integration einer Kamera (????????) Standard; Nutzung nur zur Texturierung
	\item z.T weitere Sensoren wie GNSS-Empfänger integriert $\Rightarrow$ außerdem:
	\begin{itemize}
		\item Scannende Tachymeter
		\item tachymetrisch messende Scanner
	\end{itemize}
\end{itemize}
Einteilung nach Reichweite: 
\paragraph{Nahbereich 100 bis 200m:}
Innenraumaufnahme, (?????) Management, 3D-Stadtmodelle, Industrievermessung,Monitoring...
\paragraph{Fernbereich >200m}
Monitoring, Außenraumaufnahme, Bergbau(Tagebau), Naturgefahren
\subsection{Messverfahren}
\subsubsection{Streckenmessung}
\paragraph{Impulsverfahren}
\begin{itemize}
	\item Reichweite hoch: $\leq 4km$
	\item Messrate geringer: $\leq 100 MHz$
	\item Genauigkeit geringer
\end{itemize}
\paragraph{Phasenvergleichverfahren}
\begin{itemize}
	\item Reichweite gering $\leq 200 m$
	\item Messrate höher $> 1000MHz$
	\item Genauigkeit höher
\end{itemize}
\paragraph{Kombination}
Eigenschaft auch als Kombination.
\subsubsection{Winkelmessung (Hz, V)}
\begin{itemize}
	\item Drehgebar mit Inkrementalteilung(Inkrementelle Encoder)
	\item Äquidistante Drehbewegung steuert Aussendung des Lasersignals
	\item Genauigkeit $1mgon$ bis $20mgon$
\end{itemize}
\subsubsection{Intensitätsmessung}
\begin{itemize}
	\item reflektierte Signalstärke $\Rightarrow 4D-Laserscanner$!
\end{itemize}
\subsection{Auswertestrategien}
\subsection{Registrierung und Georeferenzierung}
\subsection{Fehlerquellen}
Strahdivergenz:
\begin{itemize}
	\item Strahldurchmesser in Entfernung $Z$: $D(Z)$
	\item Strahldurchmesser beim Verlassen des Scanners: $D_0$
	\item Wellenlänge: $\lambda$
\end{itemize}
\begin{equation*}
	D(Z) = D_0 \sqrt{1 + \frac{4}{\pi} \frac{\lambda \cdot Z^2}{D_0^2}} \approx D_0 + \frac{4 \cdot \lambda}{\pi \cdot D_0} \cdot z
\end{equation*}
Beispiel: $\lambda = 660nm$ (roter Laser), $D_0 = 3mm$
\begin{table}[ht] \centering
	\begin{tabular}{|l|l|}
		\hline
		$Z$    & $D(Z)$   \\ \hline
		$10m$  & $6mm$    \\ \hline
		$100m$ & $3cm$    \\ \hline
		$1km$  & $28,3cm$ \\ \hline
	\end{tabular}
\end{table}
\subsection{Flächen- und Volumenbestimmung}
\subsection{Prüfung und Kalibrierung}
\subsubsection{Komponentenprüfung}
\paragraph{Entfernungsmessung}
\begin{itemize}
	\item Einzelmessung nicht zu realisierung
	\item Nutzung von Zielzeichen oder Kugeln
	\item Vergleich mit Soll-Werten
\end{itemize}
\paragraph{Winkelmessung}
\begin{itemize}
	\item wie bei Entfernungsmessung, aber Betrachtung von Quer- oder Höhenabweichungen $\Rightarrow$ zur Zeit systemüberprüfung (da Komponentenüberprüfung schwer realisierbar)
\end{itemize}
\subsubsection{Typische Fragestellung der Systemüberprüfung}
\begin{itemize}
	\item Messrauschen
	\item Auflösung (Detailiertheit)
	\item Kantenerkennung
	\item Oberflächenbeschaffenheit
	\item Gesamtsystem
	\item Identifizierung individueller Einflussquellen(Elementarfehler)
\end{itemize}
\subsubsection{Prüfverfahren nach VDI/VDE}
\paragraph{Antastabweichung}
\begin{itemize}
	\item Nutzung einer Kugel mit grob bekannten Radius 
	\item zehnmalige Bestimmung des Radius und des Mittelpunktes von verschiedene Position
	\item radiale Abweichung $\bar{\Delta R} = \frac{1}{n} \sum_{i=1}^{n} v_i$ pro Position
	\item mittlere Antastabweichung $S_R = \sqrt{\frac{\sum S^2_{\bar{R}}}{m}}$ mit $S^2_{\bar{R}}$ Varianz des Radius pro Person ist.
\end{itemize}
\paragraph{Abstandsabweichung}
\begin{itemize}
	\item Nutzung im Raum verteilter gleich geformter Kugeln
	\item Abstände der Kugelmittelpunkte sind bekannt 
	\item Bestimmung der Kugelmittelpunkte und der Abstände aus den Messdaten: $\bar{\Delta L} = \frac{1}{n} \sum_{i=1}^{n} \Delta L_i$ mit $\Delta L_i = L_{gemessen,i} - L_{soll,i}$ 
	\item mittlere Abstandsabweichung: $s_{\bar{l}} = \sqrt{\frac{\sum \Delta L_i^2}{n}}$
\end{itemize}
\paragraph{Ebenheitsabweichung}
\begin{itemize}
	\item Abweichung der Messungen von einer ausgleichenden Ebene
	\item Nutzung geradeförmiger Prüfkörper
\end{itemize}
pro Ebene:
\begin{gather*}
	R_E = \frac{1}{n} \sum_{i=1}^{n} V_{E,i} \
\end{gather*}
mittelere Ebenheitsabweichung:
\begin{equation*}
	S_E = \sqrt{\frac{\sum S^2_{E,j}}{m}}
\end{equation*}
$m$ ist Anzahl der Ebene, $S_{E,j}$ ist Standabweichung pro Ebene. 
\end{document}

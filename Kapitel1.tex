\documentclass[12pt]{article}
\usepackage{setspace,graphicx,amsmath,geometry,fontspec,titlesec,soul,bm,subfigure}
\titleformat{\section}[block]{\LARGE\bfseries}{\arabic{section}}{1em}{}[]
\titleformat{\subsection}[block]{\Large\bfseries\mdseries}{\arabic{section}.\arabic{subsection}}{1em}{}[]
\titleformat{\subsubsection}[block]{\normalsize\bfseries}{\arabic{subsection}-\alph{subsubsection}}{1em}{}[]
\titleformat{\paragraph}[block]{\small\bfseries}{[\arabic{paragraph}]}{1em}{}[]
\setmainfont{Times New Roman}
\renewcommand{\baselinestretch}{1.15}
\renewcommand\contentsname{Inhaltverzeichnis}
\geometry{a4paper,left=2.5cm,right=2.5cm,top=2.5cm,bottom=2.5cm}
\begin{document}
	\newpagestyle{main}{            
		\sethead{}{Kapitel 1}{}     
		\setfoot{}{\thepage}{}
		\headrule
		\footrule
			}
	\pagestyle{main}
\tableofcontents
\newpage
\section{Einführung}
\subsection{Definitionen Ingenieurgeodäsie}
Ingenieurvermessung = Ingenieurgeodäsie
\newline
Englisch: Enginierung Surveying
\newline
Frazösisch: Topometrie de genie civil
\newline
FIG: Fédération internationale des géomètres (International Association of Surveying)
\newline
\newline
\textbf{Neue Definition:}
\newline
2015 Deutsche Geodätische Kommision:
Ingenieurgeodäsie ist die Beherrschung von groß- und kleinraumigen geometrieorientierten Fragestellungen mit Schwerpunkt auf
\begin{itemize}
\item Qualität
\item Sensorik
\item Bezugssystem
\end{itemize}
für
\begin{itemize}
\item Aufnahme
\item Absteckung
\item Monitoring
\end{itemize}
\subsection{Klassische Aufgabe}
1. Aufnahmevermessung: Erfassung des Ist-Zustandes vor Bauausführung.
\newline
Beispiel: 
\begin{itemize}
\item topographische Geländeaufnahme für eine Straßentrasse;
\item Aufnahme eines Leitungsnetzes eines Chemiewerkes.
\end{itemize}
2. Projektierung (nur geometrische Anteile): Festlegung geometrischer Größen des Soll-Zustandes eines Objektes(Bauentwurf, Trassenentwurf).
\newline
Beispiel: 
\begin{itemize}
\item eine Rechnung einer Bautrasse nach den Kriterien Sicherheit
\item zurückgelegte Weg
\item Ökologie
\end{itemize}
3. Absteckung = Übertragung geometrischer Soll-Größen in die Örtlichkeit.
\newline
Beispiel: 
\begin{itemize}
\item Achsabsteckung
\item Maschinensteurung
\item Einrichtung von Maschinen
\end{itemize}
4. Abnahmevermessung = Erfassung des Ist-Zustand nach Bauausführung (unabhängige Kontrolle von Absteckung + Bauausführung und wichtig für Abrechnung).
\newline
Beispiel:
\begin{itemize}
\item Erdmassenermittlung
\item Achse Überprüfung
\item Grenzabstands Überprüfung
\end{itemize}
5. Überwachungsvermessung (Monitoring) = Vermessung zur Feststellung von Objektbewegungen und -verformungen.
\newline
Beispiel:
\begin{itemize}
\item Senkungsüberwachung im Bergbau
\item Brückenüberwachung oder Tunnel, Talsperren
\item Turbinenüberwachung
\end{itemize}
\newpage
\end{document}
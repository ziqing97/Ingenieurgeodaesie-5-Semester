\documentclass[12pt]{article}
\usepackage{setspace,graphicx,amsmath,geometry,fontspec,titlesec,soul,bm,subfigure}
\titleformat{\section}[block]{\LARGE\bfseries}{\arabic{section}}{1em}{}[]
\titleformat{\subsection}[block]{\Large\bfseries\mdseries}{\arabic{section}.\arabic{subsection}}{1em}{}[]
\titleformat{\subsubsection}[block]{\normalsize\bfseries}{\arabic{subsection}-\alph{subsubsection}}{1em}{}[]
\titleformat{\paragraph}[block]{\small\bfseries}{[\arabic{paragraph}]}{1em}{}[]
\setmainfont{Times New Roman}
\renewcommand{\baselinestretch}{1.15}
\renewcommand\contentsname{Inhaltverzeichnis}
\geometry{a4paper,left=2.5cm,right=2.5cm,top=2.5cm,bottom=2.5cm}
\begin{document}
	\newpagestyle{main}{            
		\sethead{}{Kapitel 6}{} 
		\setfoot{}{\thepage}{}
		\headrule
		\footrule
			}
	\pagestyle{main}
\tableofcontents
\newpage
\section{Tunnelvermessung und Kreisel}	
\subsection{Vermessungsaufgaben beim Tunnelbau}
\subsubsection{Absteckung}
Tunnelnetze und deren Aufbau
\begin{itemize}
\item Hauptnetz verbindet die Portale (GNSS oder Tachymeter)
\item Portalnetz: Grundlagen für Tunnelpolygon, 3-4 Punkte + Hauptnetzpunkte, tachymetrisch
\item Tunnelpolygon: (a) für den Vortrieb. (b) zur Kontrolle 
	\begin{itemize}
		\item Problem: 
		\begin{itemize}
			\item Lange einseitig angeschlossenen Polygonzug
			\item Unsicherheit des Richtungswinkel
			\item Querabweichung stiegt mit zunehmende Länge
		\end{itemize}
		\item Lösung
		\begin{itemize}
			\item Bestimmung der Richtungswinkel ohne Anschlußpunkte durch Vermessungskreisel
		\end{itemize}
	\end{itemize}
\end{itemize}
Kreiselanwendungen
\begin{itemize}
	\item Tunnelbau
	\item Bergbau
	\item Anschluss terrestische Messungen an GNSS Punkte
\end{itemize}
Altenative Lösung
\begin{itemize}
	\item Magnetische Orientierung (zu ungenau)
	\item Astronomische Orientierung (nicht möglich)
	\item GNSS Messung (nicht möglich)
\end{itemize}
\subsubsection{Abnahme und Überwachung}
\begin{itemize}
	\item Kontrollpolygon
	\item Monitoring der Umgebung (Setzung oberhalb des Tunnels)
	\item Konvergenzmessungen (Stabilitätsprüfung des Tunnels)
\end{itemize}
\subsection{Vermessungskreisel}

\end{document}

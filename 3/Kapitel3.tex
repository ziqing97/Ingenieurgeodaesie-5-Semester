\documentclass[12pt]{article}
\usepackage{setspace,graphicx,amsmath,geometry,fontspec,titlesec,soul,bm,subfigure}
\titleformat{\section}[block]{\LARGE\bfseries}{\arabic{section}}{1em}{}[]
\titleformat{\subsection}[block]{\Large\bfseries\mdseries}{\arabic{section}.\arabic{subsection}}{1em}{}[]
\titleformat{\subsubsection}[block]{\normalsize\bfseries}{\arabic{subsection}-\alph{subsubsection}}{1em}{}[]
\titleformat{\paragraph}[block]{\small\bfseries}{[\arabic{paragraph}]}{1em}{}[]
\setmainfont{Times New Roman}
\renewcommand{\baselinestretch}{1.15}
\renewcommand\contentsname{Inhaltverzeichnis}
\geometry{a4paper,left=2.5cm,right=2.5cm,top=2.5cm,bottom=2.5cm}
\begin{document}
	\newpagestyle{main}{            
		\sethead{}{Kapitel 3}{} 
		\setfoot{}{\thepage}{}
		\headrule
		\footrule
			}
	\pagestyle{main}
\tableofcontents
\newpage
\section{Qualität in der Ingenieurgeodäsie}
\subsection{Qualitätsdefinition und -modelle}
\subsection{Qualität im Bauprozess}
Hauptqualitätsparameter sind Toleranz, die dem Qualitätsmerkmal Korrektheit zugeordnet werden können.(manchmal wird aber von Genauigkeit gesprochen) 
\begin{itemize}
\item Nennenmaß $N$ / Sollmaß $S$
\item Höchstmaß $G_o$
\item Mindestmaß $G_u$
\item Mittemaß $C = \frac{1}{2} (G_o + G_u)$ ($=N$ wenn symmetrisch)
\item Obere Abmaß: $A_o= G_o - N$
\item Untere Abmaß: $A_u = G_u - N$
\item Toleranz: $T = G_o - G_u = A_u - A_o$
\end{itemize}
Die Toleranz ist der Spielerraum, der der allem am Bau Beteiligten zur Behandlung zufälliger Abweichungen zur Verfügungen steht.\newline
Die Toleranz bezieht sich auf eine Bezugstemperatur und verändert sich bei Abweichungen hiervon.\newline
\newline
Nach Bau / Realisierung: \newline
Istmaß $I$ ist Ergebnis einer Messung nach Bauausführung, darf nicht außerhalb der Toleranz liegen. Istabmaß $A_i = I - N$.\newline
Forderung: $G_u \leq I \leq G_o$ \newline
\newline
Toleranzfortpflanzung / Toleranzkette.
\begin{itemize}
\item  $T_i$: Einzeltoleranzen (Eingangsgrößen)
\item $T_G$: Gesamttoleranz (Ausgang)
\item $x_i$: Einflussgrößen
\item $y$: Zielgrößen
\end{itemize}
2 Möglichkeiten\newline
a. quadratische Fortpflanzung
\begin{gather*}
T_G = \sqrt{\sum_{i = 1}^{n} (\frac{\partial y}{\partial x_i})^2 T_i^2} \\
T_G = \sqrt{\sum_{i = 1}^{n} T_i^2} \quad   (vereinfacht)
\end{gather*}
Wenn Abweichung klein und zufällig, vor allem Bauwesen.\newline
\newline
b. lineare Fortpflanzung\newline
\begin{gather*}
T_G = \sum_{i = 1}^{n} T_i
\end{gather*}
Wenn Abweichung relative groß und Tendenz zur Systematisch besteht, häufig im Maschinenbau. Abschätzung pessimitisch.\newline
\newline
Immer gilt: $T_{G,linear} > T_{G,quadr}$.\newline
Über Toleranz: hinausgesehende Pasmik werden im Bauwesen kaum genutzt. Hierzu nur vergangene und aktuelle Forschungsprojekt. 
\subsection{Qualität im Ingenieurgeodätischen Prozess}
Wesentlicher Qualitätsmerkmal ist die Genauigkeit. (meist mit Standardabweichung als Parameter) \newline
\newline
Wahre Abweichung: $\eta = x-\tilde{x} = \Delta + \varepsilon$ \newline
Zufällige Abweichung $E(\varepsilon) = 0 \longrightarrow$  Prezision/Widerholstandardabweichung \newline
Systemmatische Abweichung: $E(\Delta) \neq 0 \longrightarrow$ Richtigkeit, Korrektheit, Vergleichsabweichung. \newline
\newline
In der Geodäsie werden systemmatische Abweichung häufig in bekannte und unbekannte systemmatische Abweichung unterteilt: 
\begin{itemize}
\item bekannte Abweichung werden eliminiert (z.B. durch Kalibrierung)
\item unbekannte Abweichung werden durch Messordnung randomisiert und als zufällig behandelt.
\end{itemize}
Widerholstandardabweichung (unter identischen Bedingungen ermittelte) (Innere Genauigkeit):
\begin{gather*}
\sigma_{\varepsilon} = \frac{1}{n-1} \sqrt{\sum_{i=1}^{n} \varepsilon^2_i}
\end{gather*}
Vergleichstandardabweichung (unter möglichst unterschiedliche Bedingungen) (Äußere Genauigkeit):
\begin{gather*}
\sigma = \sqrt{\sigma_{\varepsilon}^2 + \sigma_{\Delta}^2}
\end{gather*}
Konfidenzbereich:
\begin{equation*}
p(\bar{x} - k_{\sigma} \leq \mu \leq \bar{x} + k_{\sigma})
\end{equation*}
$k$ ist im Prinzip das Quantil der Verteilungsfunktion. Bei Normalverteilung gilt:
\begin{itemize}
\item $k \approx 1 \longrightarrow \alpha = 32\% \quad p=68\%$
\item $k \approx 2 \longrightarrow \alpha = 5\% \quad p=95\%$
\item $k \approx 3 \longrightarrow \alpha = 0,03\% \quad p=99,97\%$
\end{itemize}
Qualitätsmerkmale Genauigkeit, Zuverlässigkeit sind für geodätische Netze bekannt; für Prozesse ist Entwicklungsbedarf
\subsection{Zusammenhang zwischen Standardabweichungen und Toleranz}
\begin{gather*}
T = \sqrt{T_M + T_A}
\end{gather*}
 $T_A$: Ausführungstoleranz $= \sqrt{T_{Fertigung}^2 + T_{Montage}^2}$ \newline
 $T_M$: Vermessungstoleranz,häufig als Angabe in Prozent $P$ an der Gesamttoleranz $T \rightarrow T_A = (1-P)T$ \newline
 Qualität Toleranzfortpflanzung:
 \begin{align*}
 T^2 = & T_M^2 + T_A^2 \\
 = & T_M^2 + (1-P)^2 T^2 \\
 T_M^2 = & T^2 - T_A^2 \\
 = & T^2 (1-(1-P)^2)	\\
 T_M = & T \sqrt{2P-P^2}
 \end{align*}
 Beispiel: $P = 10 \% \longrightarrow T_M = 0,44T$ \newline
 Prinzipiell darf kein Messwert außerhalb der Toleranz liegen. \newline
 Anhährung an der Toleranz über Konfidenzbereich.
 \begin{equation*}
 T_M = C_O(obere) - C_U(untere)
 \end{equation*}
 C: Grenzen des Konfidenzbereiches.
 \begin{gather*}
 P(C_U \leq \mu \leq C_O) = 1-\alpha \\
 P(\bar{x} - k \cdot \sigma \leq \mu \leq \bar{x} + k \cdot \sigma) = 1-\alpha \\
 T_M = C_O - C_U = 2 \cdot k \cdot \sigma
 \end{gather*}
 $k$: Quantil der Verteilungsfunktion i.d.R wird Normalverteilung vorausgesetzt.
 \begin{gather*}
 1-\alpha = 95\% \longrightarrow k=1,96 \\
 1-\alpha = 99,75\% \longrightarrow k \approx 3 \\
 \sigma = \frac{T_M}{2 \cdot k}
 \end{gather*}
 Beispiel: $k=2$, $T_M = 0,44T \longrightarrow \sigma = 0,11T$
 \begin{table}[ht] \centering
 	\begin{tabular}{|l|l|l|l|}
 		\hline
 		P   &  $\frac{T_M}{T}$    &  $\frac{\sigma}{T} (k=2)$    &   $\frac{\sigma}{T} (k=3)$   \\ \hline
 		0,1 & 0,44 & 0,11 & 0,07 \\ \hline
 		0,2 & 0,60 & 0,15 & 0,10 \\ \hline
 		0,3 & 0,7  & 0,18 & 0,12 \\ \hline
 		0,5 & 0,87 & 0,22 & 0,14 \\ \hline
 		0,9 & 0,99 & 0,25 & 0,16 \\ \hline
 	\end{tabular}
 \end{table}
\newline
In der Praxis: $P$ zwischen $0,3$ und $0,5$, $\sigma = \frac{1}{5}T\ (k=2)$, $\sigma = \frac{1}{8}T\ (k=3)$ \newline
\newline
Beispiel:\newline
Lagetolerenz einer Straßenachse: $5\ cm \longrightarrow$ Verwendung Faustformel für $\alpha = 5 \%$
\begin{equation*}
\sigma = \frac{1}{5} T  = 1\ cm \quad 
\end{equation*}
$P$ zwischen $[0,3\quad 0,5]$, $k=2$ \newline
Tachymeter mittelere Genauigkeit: 
\begin{gather*}
\sigma_s = \sqrt{(5mm)^2 + (5ppm)^2} \\
\sigma_r = 1mgon\\
\sigma_w=\sqrt{2} \sigma_r\\
\sigma_p=\sqrt{\sigma_Q^2 + \sigma_L^2}
\end{gather*}
\begin{itemize}
\item Längst. $\sigma_L \rightarrow$ Strecke \\
\item Querst. $\sigma_Q \rightarrow$ Winkel
\end{itemize}
\begin{gather*}
\sigma_L = \sigma_s\\
\sigma_Q = \sigma_w \frac{s}{\rho}
\end{gather*}
\begin{table}[ht] \centering
	\begin{tabular}{|l|l|l|}
		\hline
		s & 200m  & 400m   \\ \hline
		$\sigma_L$ & 5,1mm & 5,4mm  \\ \hline
		$\sigma_Q$ & 4,4mm & 8,8mm  \\ \hline
		$\sigma_p$ & 6,7mm & 10,3mm \\ \hline
	\end{tabular}
\end{table}
\newline
Standardabweichung eingehalten; Tachymeter mittele Genauigkeit kann gesetzt werden. 
\subsection{Messunsicherheit nach GUM}
GUM: Guide for the expression of uncentaining in measurment. \newline
Unsicherheit: 
\begin{itemize}
\item Verwendung als Kriterium analog zu Genauigkeit
\item Verwendung als Parameter; hiervon sind verschiedene definiert
\end{itemize}
Type A für zufällige Einflüsse (wie unter wiederholbar), Type B für (rest) systematische Einflüsse(über Monte Carlo Simulation). \newline
Kombinierte Unsicherheit
\begin{gather*}
U_C = \sqrt{U_A^2 + U_B^2} \\
\end{gather*}
Formel gilt für nicht korreliert Fehlerquellen, aber erweiterbar, gewisse Analogie zum Elementarfehlermodell.	
\begin{equation*}
U = k \cdot U_C
\end{equation*}
Kennwert der einen Bereich um den Messwerte charakterisiert, der zu einem Großteil die zum Messwert gehörige Verteilung beinhaltet.\newline
Eine direkte Verbindung zum Konfidenzintervall ist nicht gegeben, da Normalverteilung nicht vorausgesetzten werden kann.\newline
Erwartungsfaktor $k$: Zahlenfaktor zwischen kombinierte und erweiterte Unsicherheit, in der Regel zwischen 2 und 3(bei Normalverteilung würde das $\alpha = 5\%	$ bzw. $\alpha =0,3\%$ bedeuten), analog zum Konfidenzbereich, aber ohne statistische Aussage. 
\end{document}
